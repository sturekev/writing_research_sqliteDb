\documentclass[11pt]{sigplanconf}

\usepackage{hyperref}  % need this for links
\usepackage{graphicx}  % need this for images
\usepackage{listings}  % need this for code

% Preamble
% Metadata
\title{Writing in the Major Lab (CS~296)}
\authorinfo{Roman Yasinovskyy}{Luther College}{roman@luther.edu}
\date{\today}
\copyrightdoi{}

\begin{document}

\maketitle

\tableofcontents

\begin{abstract}

    This document describes basic tools and elements of \LaTeX one needs to start working on a paper. Many elements of the style are specified in the SIGPLAN class file\cite{SIGPLANL45:online} and \LaTeX fundamentals are described in the \textit{Learn \LaTeX in 30 minutes} on Overleaf\cite{LearnLaT95:online}.

\end{abstract}

\section{Tools}

\subsection{tex-live}

You are going to need \texttt{tex-live} package to work with \LaTeX. \texttt{tex-live} is a multi-platform \TeX document production system \cite{TeXLiveT67:online}. It comes packed with various tools and you may use command line to compile your file, but you can also use GUI tools like \textit{TeXShop} (macOS), \textit{Kile} (multi-platform), \textit{Texmaker} (multi-platform), or \textit{LaTeX Workshop} for \textit{VS Code}.

\begin{verbatim}
    sudo apt install tex-live
\end{verbatim}

\subsection{Kile}

\textit{Kile} is an IDE for \LaTeX that allows you to compile, convert, and preview your document\cite{KileanIn6:online}.

\begin{verbatim}
    sudo apt install kile
\end{verbatim}

\subsection{LaTeX Workshop for VS Code}

If you prefer \textit{VS Code} to write code, install the \textit{LaTeX Workshop} extension to write your \LaTeX, build (compile) it, and generate (preview) the resulting PDF.

\subsection{Lucid chart}

While professional tools like \textit{OmniGraffle} (macOS) or \textit{Visio} (Windows) are usually used to create diagrams, \textit{Lucid chart}\cite{OnlineDi2:online} should be sufficient for the purposes of this paper and it is free. You should not include photos in your paper but rather draw diagrams and generate charts\footnote{Use \textit{Excel} or \textit{Spreadsheets} for charts}.

\section{Structure}

The main goal of this course if for you write a scientific paper while using proper tools and methods. Your paper is going to be a survey/review of existing sources and should not exceed 7 pages.

Sections of the paper should include at least the following sections:

\begin{itemize}
    \item Introduction
    \item History of the subject
    \item Prominent features
    \item Conclusion
    \item References
\end{itemize}

\section{Timeline}

You are expected to stick to the schedule specified on KATIE\ref{tbl:schedule}.

\section{Advanced elements}

\subsection{Math}

Your paper may include mathematical formulas. They can appear \textit{inline} (e.g. \begin{math}i^2=-1\end{math} or $E=mc^2$) or in \textit{display} mode.

\begin{equation}
    F = G \frac{m_1 m_2}{r^2}
\end{equation}

or

$$a^2 + b^2 = c^2$$

\subsection{Code}

An easy way to include code is to use package \texttt{listings} and have your code in a separate file. Other options (e.g. package \texttt{minted}) are acceptable too but may require additional tools.

\lstinputlisting[language=Python]{panda.py}

You can also include code in the body of your document.

\begin{lstlisting}[language=Python, numbers=left, numberstyle=\tiny]
    def hello():
        print("Hello, Panda")
    
\end{lstlisting}

\subsection{Image}

An image\ref{fig:panda1} or a chart can be inserted into the document.

\begin{figure}[!htbp]
    \includegraphics[width=0.4\textwidth]{panda.png}
    \caption{Panda}
    \label{fig:panda1}

\end{figure}

\subsection{Fancy text}

Text \reflectbox{reflected} horizontally.
Text \scalebox{1}[-1]{reflected} vertically.

\bibliographystyle{unsrtnat}
\bibliography{latex_sources}

\section*{Appendix A}

\begin{table}[!htbp]
    \centering
    \begin{tabular}{|l|c|c|}
        \hline
        Task                 & Week & Points \\
        \hline
        \LaTeX seminar       & 1    & 5      \\
        Select a topic       & 1    & 10     \\
        Meet the librarian   & 2    & 5      \\
        Identify the sources & 2    & 10     \\
        Outline              & 3    & 10     \\
        First draft          & 4    & 20     \\
        Meet the instructor  & 5    & 0      \\
        Final draft          & 7    & 20     \\
        Presentation         & 8    & 10     \\
        Full paper           & 8    & 10     \\
        \hline
    \end{tabular}
    \caption{Tentative schedule}
    \label{tbl:schedule}
\end{table}

\end{document}
