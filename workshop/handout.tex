%% See examples in the acmart-primary.zip/samples for details
\documentclass[nonacm]{acmart}

%% Packages %%
\usepackage{listings}  % code

%% Preamble %%
%% Metadata %%
\title{Writing in the Major Lab (CS\ 296)}

\author{Roman Yasinovskyy}
\email{roman@luther.edu}
\affiliation{
    \institution{Luther College}
    \streetaddress{700 College Dr}
    \city{Decorah}
    \state{Iowa}
    \country{United States}
    \postcode{52101}
}

\date{\today}

%% Document body %%
\begin{document}

%% Abstract. Must be before \maketitle is called %%
\begin{abstract}

    This document describes basic tools and elements of \LaTeX\ one needs to start
    working on a paper. \LaTeX\ fundamentals are described in
    \cite{OverleafTutorial} and \cite{oetiker1995not}.

\end{abstract}

\maketitle

\tableofcontents

\section{Tools}

\subsection{tex-live}

You are going to need \texttt{tex-live} package to work with \LaTeX. It is a
multi-platform \TeX\ document production system\cite{TeXLive} that comes packed
with various tools you may use to compile your file. You can also use GUI tools
like \emph{TeXShop} (macOS), \emph{Kile} (multi-platform), \emph{Texmaker}
(multi-platform), or \emph{LaTeX Workshop} for \emph{VS Code}.

\begin{lstlisting}[caption=Installing tex-live, captionpos=b, frame=trbl]
sudo apt install tex-live
\end{lstlisting}

\subsection{Kile}

\emph{Kile} is an IDE for \LaTeX\ that allows you to compile, convert, and preview your document.

\begin{lstlisting}[caption=Installing Kile, captionpos=b, frame=trbl]
sudo apt install kile
\end{lstlisting}

\subsection{LaTeX Workshop for VS Code}

If you prefer \emph{VS Code} to write code, install the \emph{LaTeX Workshop}
extension to write your \LaTeX, build (compile) it, and generate (preview) the
resulting PDF.

\subsection{Lucid chart}

While professional tools like \emph{OmniGraffle} (macOS) or \emph{Visio}
(Windows) are usually used to create diagrams, \emph{Lucid chart} should be
sufficient for the purposes of this paper and it is free. You should not
include photos in your paper but rather draw diagrams and generate
charts\footnote{Use \emph{Excel} or \emph{Spreadsheets} for charts}.

\section{Structure}

The main goal of this course if for you write a scientific paper while using
proper tools and methods. Your paper is going to be a survey/review of existing
sources and should not exceed 7 pages. You should use class \emph{article}.

The paper should include at least the following sections:

\begin{itemize}
    \item Introduction
    \item History of the subject
    \item Prominent features
    \item Conclusion
    \item References
\end{itemize}

\section{Timeline}

You are expected to stick to the schedule specified on KATIE (see
Table~\ref{tbl:schedule} in the Appendix).

\section{Advanced elements}

\subsection{Math}

Your paper may include mathematical formulas. They can appear \emph{inline}
(e.g. \begin{math}i^2=-1\end{math} or \(E=mc^2\)) or in \emph{display} mode.

%% Numbered equation %%
\begin{equation}
    F = G \frac{m_1 m_2}{r^2}
\end{equation}

or

\[a^2 + b^2 = c^2\]

\subsection{Code}

An easy way to include code is to use package \texttt{listings} and have your
code in a separate file. Other options (e.g. package \texttt{minted}) are
acceptable too but may require additional tools.

\lstinputlisting[language=Python, numbers=left, numberstyle=\tiny, caption=\texttt{hello} from file, captionpos=b, frame=trbl]{panda.py}

\noindent You can also include code in the body of your document.

\begin{lstlisting}[language=Python, caption=\texttt{hello} inline and with different options, captionpos=b, frame=trbl, showstringspaces=false]
def hello():
    print("Hello, Panda")
    
\end{lstlisting}

\subsection{Image}

An image (see Figure~\ref{fig:panda}) or a chart can be inserted into the
document.

\begin{figure}[!htbp]
    \centering
    \includegraphics[width=0.4\textwidth]{panda.png}
    \Description{Panda}
    \caption{Panda}
    \label{fig:panda}
\end{figure}

\subsection{Fancy text}

Text \reflectbox{reflected} horizontally. Text \scalebox{1}[-1]{reflected}
vertically.

%% Bibliography. Must be just before the end of the document %%
\bibliographystyle{ACM-Reference-Format}
\bibliography{sources}

\clearpage
\section*{Appendix A}

\begin{table}[!htbp]
    \centering
    \caption{Tentative schedule}
    \label{tbl:schedule}
    \begin{tabular}{l c c}
        \hline
        Task                 & Week & Points \\
        \hline
        \LaTeX seminar       & 1    & 5      \\
        Select a topic       & 1    & 10     \\
        Meet the librarian   & 2    & 5      \\
        Identify the sources & 2    & 10     \\
        Outline              & 3    & 10     \\
        First draft          & 4    & 20     \\
        Meet the instructor  & 5    & 0      \\
        Final draft          & 7    & 20     \\
        Presentation         & 8    & 20     \\
        \hline
    \end{tabular}
\end{table}

\end{document}